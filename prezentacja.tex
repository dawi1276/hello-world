\documentclass{beamer}
\usepackage[utf8]{inputenc} 
\title{Dzsonans Poznawczy}
\author{Dawid Szczepkowski}
\institute{UWM}
\date{\today}
\usepackage{amsfonts}
\usepackage[MeX]{polski}
\begin{document}
\frame{\titlepage}



\begin{frame}
\frametitle{Spis Treści}
\tableofcontents
\end{frame}



\section{historia}
\begin{frame}{Historia}
\begin{itemize}
\item{Teoria dysonansu poznawczego stworzona została w pierwotnej wersji przez Leona Festingera w 1957. Rozwijana dalej między innymi przez Andrzeja Malewskiego, Jacka Brehma, uczniów Festingera Elliota Aronsona, Merrilla Carlsmitha i in.


W rejonach epicentrum szerzyły się pogłoski o rychło nadchodzącej pomocy. W rejonach gdzie wstrząsy były odczuwalne, ale niegroźne, rozpowszechniały się pełne obaw plotki o nadchodzących kolejnych nieszczęściach.
Pomysł teorii. Festinger twierdził, że wpadł na pomysł swojej teorii po tym, jak dowiedział się o trzęsieniu ziemi z 1934 r. w Indiach. Bezpośrednio po kataklizmie w okolicznych wioskach (których mieszkańcy nie zostali dotknięci szkodami, lecz odczuwali dość silne wstrząsy) pojawiły się plotki o rychłym nadejściu kolejnych katastrof. W tym samym czasie wśród mieszkańców dotkliwie poszkodowanych (epicentrum) rozchodziły się plotki o rychłym nadejściu pomocy rządowej.}

\end{itemize}
\end{frame}

\section{główne założenie}
\begin{frame}{Główne założenie}
\begin{itemize}
\item Podstawowym założeniem w teorii dysonansu poznawczego jest twierdzenie o pojawianiu się nieprzyjemnego napięcia psychicznego wtedy, gdy u danej osoby występują (lub dotrą do niej) sprzeczne elementy poznawcze (mogą to być twierdzenia, przemyślenia, postawy, informacje, oceny, zachowania itp.).

Dysonans poznawczy można nazwać popędem, bowiem jego wpływ na zachowanie jest taki, jak podstawowych popędów człowieka (np. głodu, pragnienia, bólu): wywołuje ogólną mobilizację organizmu, motywuje do zachowań, których celem jest zmniejszenia napięcia, oraz wywołuje antycypacyjne unikanie (czyli uczenie się reakcji na bodźce skojarzone z pojawieniem się dysonansu).
\end{itemize}
\end{frame}






\section{przewidywania}
\begin{frame}{przewidywania}
\begin{itemize}
\item
Jedna z najczęstszych rozbieżności występuje między postawami i wartościami z jednej strony, a zachowaniami z drugiej. Gdy człowiek uświadamia sobie, że postępuje niezgodnie z własnymi przekonaniami, odczuwa przymus przywrócenia harmonii.
\end{itemize}
\end{frame}







\section{Dysonans poznawczy a myślenie racjonalne}
\begin{frame}{Dysonans poznawczy a myślenie racjonalne}
\begin{itemize}
\item
Dysonans poznawczy może upośledzać myślenie racjonalne. Antycypacja napięcia dysonansowego prowadzi do unikania myślenia o pewnych kwestiach, bowiem mogłyby to doprowadzić do pojawienia się „nieprzyjemnych” wniosków. Na przykład palacze zwykle są przekonani, że palą mniej niż rzeczywiście palą, a nieleczący się alkoholicy uważają, że nie mają rzeczywistego problemu z piciem.

Istnieje także bardziej subtelny wpływ dysonansu poznawczego na zdolności do racjonalnego myślenia.
\end{itemize}
\end{frame}




\section{Dysonans poznawczy a uzasadnienie wysiłku}
\begin{frame}{Dysonans poznawczy a uzasadnienie wysiłku}
\begin{itemize}
\item
Jeszcze innym obszarem zachowań człowieka, w którym teoria dysonansu poznawczego tworzy oryginalne przewidywania, jest dążenie ludzi do określonych celów. Teoria przewiduje, że im więcej wkłada się wysiłku w osiągnięcie jakiegoś celu, tym wyżej będzie się go cenić. Wywołuje bowiem silny dysonans konstatacja, że wiele poświęciliśmy dla osiągnięcia celu, który nie był tego wart. Ten sam cel osiągnięty bez wysiłku powinien być oceniany jako mniej wartościowy.
\end{itemize}
\end{frame}








\section{Techniki manipulacji}
\begin{frame}{Techniki manipulacji}
\begin{itemize}
\item
istnieje kilka technik manipulacyjnych opracowanych w oparciu o teorię dysonansu. Najpopularniejsze to: stopa w drzwi (występująca też w postaci zmasowanej – „dwie stopy w drzwiach”), technika scenariusza, niska piłka, drzwiami w twarz.

Wszystkie one wykorzystują napięcie dysonansowe, które wiąże się bądź to z odmawianiem osobom, które nas o coś proszą (drzwiami w twarz), bądź to wywołują zaangażowanie (stopa w drzwi, niska piłka), bądź też stwarzają iluzję istnienia okazji, z której nie warto rezygnować (technika scenariusza).

Zaangażowanie się w działania również ma moc wywoływania dysonansu. Po drobnych poświęceniach jesteśmy coraz to mniej skłonni do wycofania się z tego działania. Budziłoby to dysonans „tyle poświęciłem, a okazuje się to niewarte moich wysiłków? Skąd, to jest warte dalszych wysiłków!”, „Jeśli powiedziałeś A, musisz powiedzieć B” – i zaangażowanie rośnie.
\end{itemize}
\end{frame}




\section{rozwój teorii i kontrowersje}
\begin{frame}{rozwój teorii i kontrowersje}
\begin{itemize}
\item
Festinger pierwotnie definiował dysonans jako niezgodność „między dwoma elementami poznawczymi”, z milczącym założeniem, że nie jest istotne, czego dotyczą owe „elementy poznawcze”. Do wzbudzenia dysonansu wystarczała sama niezgodność poznawcza.

Dzisiaj wiadomo, że tak nie jest. Ludzie mają różne elementy poznawcze, które logicznie nie są ze sobą zgodne, a mimo to nie wzbudzają dysonansu. Na przykład ktoś może uważać, że jest punktualną osobą, ale spóźnia się zwykle do kościoła itp. Nie chodzi więc tu o niezgodność w sensie logicznym. Dysonans wzbudzany jest także przez twierdzenia, które w sensie logicznym nie muszą być w sprzeczności (nie są w logicznej sprzeczności twierdzenia: „Palenie jest szkodliwe” i „Palę papierosy”).

Obecnie definiuje się „niezgodność poznawczą” nie w sensie logicznym, ale psychologicznym – czyli jest to istnienie takich twierdzeń czy elementów poznawczych na temat danej kwestii, z których wynikają sprzeczne zachowania lub oceny.

Szybko okazało się także, że dysonans wzbudzany jest przede wszystkim wtedy, gdy niezgodność dotyczy w jakimś stopniu obrazu własnej osoby (jako pierwszy badania na ten temat prowadził Andrzej Malewski).

Doniosłe znaczenie ma tu samoocena. Przede wszystkim osoby o wysokiej samoocenie (i te którym bardzo zależy na wysokiej samoocenie) przeżywają silny dysonans, gdy uczynią rzecz złą lub głupią. Dysonans pojawia się więc w takich sytuacjach, które naruszają samoocenę i dobre samopoczucie.
\end{itemize}
\end{frame}
\end{document}