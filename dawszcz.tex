\documentclass[a4paper,12pt]{article}
\usepackage[MeX]{polski}
\usepackage[utf8]{inputenc}
\usepackage{graphicx}
%opening
\title{Mazda MX-5}
\author{Dawid Szczepkowski}

\begin{document}

\maketitle

\begin{abstract}

\end{abstract}
Mazda MX-5 --- samochód osobowy typu roadster produkowany przez japońską markę Mazda od 1989 roku. W USA pojazd występował pod nazwą Miata, a w Japonii jako Eunos Roadster. Od 2014 roku produkowana jest czwarta generacja modelu.

MX-5 została wprowadzona, gdy produkcja małych roadsterów praktycznie dobiegała końca. Alfa Romeo Spider była wtedy jedynym porównywalnym modelem dostępnym w sprzedaży – podczas gdy dekadę wcześniej dostępna była cała gama samochodów tego typu – jak MG B, Triumph TR7, Triumph Spitfire czy Fiat Spider.

Nadwozie jest konwencjonalną, choć stosunkowo lekką, samonośną konstrukcją, z demontowalną przednią i tylną ramą dodatkową. MX-5 wyposażono także we wspornik o nazwie marketingowej Powerplant Frame (PPF), który łączy silnik z dyferencjałem, ograniczając naprężenia i przekładając się na dobrze reagujące prowadzenie. Niektóre MX-5 są wyposażone w szperę oraz ABS. System kontroli trakcji jest opcjonalnie dostępny w modelu NC. Wcześniejsze wersje ważyły niewiele ponad tonę, z silnikami o mocy rzędu 116 KM; późniejsze miały większą masę oraz mocniejsze jednostki napędowe.

Dzięki balansowi wagi przód/tył bliskiemu 50:50, samochód ma niemal neutralne prowadzenie. Wprowadzanie samochodu w nadsterowność jest proste i łatwe do opanowania, co czyni MX-5 popularnym wyborem do amatorskich i fabrycznych wyścigów.

MX-5 zdobyła wiele nagród, takich jak Samochód Roku 1989 oraz 2005 magazynu Wheels Magazine; ,,najlepszy samochód sportowy lat 90.'' i ,,jeden z dziesięciu najlepszych samochodów sportowych wszech czasów'' od Sports Car International.


\begin{figure}
\includegraphics[]{miata.jpg}
\caption{Mazda MX-5 MK-1}\label{fig:Mazda}
\end{figure}


\begin{table}
\begin{tabular}{lc}
\hline
\textbf{rodzaj}&\textbf{dane}\\
\hline
producent&Mazda\\
Zaprezentowany&Chicago 10 lutego 1989\\
Okres produkcji&1989 - 1998\\
Miejsce produkcji&Hiroszima\\
Następca&	Mazda MX-5 II\\
\hline
\end{tabular}
\caption{tabela Mazdy Miaty}
\end{table}


\section{}

\end{document}